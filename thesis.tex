\documentclass[12pt, oneside, final]{lib/ucthesis}
\def\ssp{\def\baselinestretch{1.0}\large\normalsize}
\ssp

\usepackage{subfiles}
\usepackage{cite}
\usepackage{array}
\newcolumntype{P}[1]{>{\centering\arraybackslash}p{#1}}
\usepackage{amsmath}    % need for sub equations
\usepackage{amsfonts}
\usepackage{amssymb}
\usepackage{graphicx}   % need for figures
\usepackage{subcaption}
\usepackage{url}
%\usepackage{algorithmic}
\usepackage{algorithm}
\usepackage{algpseudocode}
\usepackage{color}
\usepackage{mathtools}    % need for sub equations
\usepackage{cancel}
\usepackage{bm}
%\usepackage[ruled,vlined,titlenotnumbered,linesnumbered]{algorithm2e}
\usepackage{float}
\usepackage{enumitem}
\usepackage{booktabs}
%\usepackage{gensymb}
%\usepackage{fixltx2e}
\usepackage{array}
\usepackage{bm}
%\usepackage{subfig}

\pdfinfo{
       /Title      (U. C. Berkeley Dissertation)
       /Author     (Qie Hu)
       /Keywords   ()
    }
% This gives you control over how far down in the hierarchy the
% table of contents will print. I use 2.
\setcounter{tocdepth}{1}
\setcounter{secnumdepth}{3}
\setlength{\parindent}{5ex}

% \include{chapters/preamble}

% \graphicspath{{resources/}}

%\newcommand\degrees{\ensuremath{^\circ}}
\newcommand{\tab}{\hspace{5mm}}
\newcommand{\blankpage}{\clearpage ~ \newpage}

\newtheorem{lem}{Lemma}
\newtheorem{remark}{Remark}
\newtheorem{theorem}{Theorem}
\newtheorem{proposition}{Proposition}
\newtheorem{result}{Result}
\newtheorem{claim}{Claim}
\newtheorem{corollary}{Corollary}
\newtheorem{definition}{Definition}[section]
\newtheorem{assumption}{Assumption}[section]

\DeclareMathOperator*{\argmin}{\arg\!\min}
\DeclareMathOperator*{\argmax}{\arg\!\max}

% SE for linear systems
\usepackage{epsfig}
\usepackage{marvosym}
\usepackage{soul}
\usepackage{xcolor}
\usepackage{epstopdf}
\usepackage{cancel}
\usepackage{pifont}
\usepackage{url}
%\usepackage{amsthm}

% SE for nonlinear power systems
\usepackage{steinmetz}

\newcommand*{\dt}[1]{%
\newcommand{\pf} {{\textit{Proof :}}}
%\newcommand{\argmax}{\operatornamewithlimits{argmax}}
\allowdisplaybreaks
\newcommand{\norm}[1]{\left\lVert#1\right\rVert}
\newcommand{\abs}[1]{\left\lvert#1\right\rvert}
 \accentset{\mbox{\large\bfseries .}}{#1}}
%\renewcommand{\baselinestretch}{0.96}
%\newcommand*{\suchthat}{\;\ifnum\currentgrouptype=16 \middle\fi|\;}








\pagestyle{myheadings}
% ========================================= DOCUMENT
\begin{document}

% Declarations for Front Matter

% TITLE
\title{Toward a Reliable Power Grid: \\ Frequency Regulation from Buildings and Secure State Estimation for the Grid}
%\title{Toward a Smart and Secure Power Grid: \\ Frequency Regulation from Buildings and Secure State Estimation for the Grid}
% Note that this must be exactly as it appears in University records.
\author{Qie Hu}

% PREVIOUS DEGREES
% Put each previous degree on its own line in the following format:
\prevdegrees{} % Optional 

% DATE OF GRADUATION
% This text will appear on the title page
% Note that degrees are only granted in Fall and Spring at Berkeley.
% This text will appear on the abstract page.
% For Berkeley, it should be identical to the graduation month.
\degreeyear{2017}
\degreemonth{Fall}
\defensemonth{Fall}
\defenseyear{2017}

\degree{Doctor of Philosophy}
% COMMITTEE MEMBERS
% You can have up to 5 members listed separately. 
%After that, you throw them all into the "other members" category.
\numberofmembers{3} 
	\chair{Professor Claire Tomlin} 
	\othermemberA{Professor Murat Arcak}
	\othermemberB{Professor Anil Aswani} 
	
% DEPARTMENT/DEGREE PROGRAM
%Your Department. Make sure this is the department and/or program
% name that you are enrolled in...
\field{Engineering - Electrical Engineering and Computer Sciences} 

% CAMPUS NAME
% Your UC Campus, e.g., "Berkeley"
% Note that if you are not at Berkeley, you may have to modify the\vspace{12pt}
% ucthesis.cls to change the wording on the Title page.
\campus{Berkeley}

\begin{frontmatter} 
\maketitle
%\approvalpage
\copyrightpage
\abstract
This thesis presents progress in overcoming two challenges in achieving a reliable electric power system: frequency regulation and secure state estimation against cyber attacks.

Taking frequency regulation first, this is a type of reserve used by the grid operator to control the grid frequency around its nominal value and therefore, achieve a balance between electricity generation and consumption, which is required at all times in order to maintain the normal operation of the power system.
%a balance between electricity generation and consumption at all times is one of the conditions required to maintain the normal operation of the electric power system.
%Frequency regulation is a type of reserve used by the grid operator to control the grid frequency around its nominal value and thus, achieve the balance between generation and consumption.
However, the increased penetration of renewable energy sources has aggravated the volatility and uncertainty of electricity generation, which leads to a greater demand for frequency regulation reserves.
This thesis explores the feasibility of using commercial buildings for frequency regulation through experimental demonstration in an occupied building.

On the other hand, instruments such as phasor measurement units that communicate over wireless networks to achieve better efficiency of the power system are becoming pervasive, especially under the smart grid initiatives. 
However, these communication networks are vulnerable to cyber attacks that can be erratic and difficult to predict.
To add to the challenge, the dynamics of the power system cannot be approximated by a linear model when it's under severe disturbances.
This thesis first develops a secure state estimation method for linear dynamical systems under sensor attack, and then extends it to two classes of nonlinear systems which is applied to the nonlinear power system.
Both estimation methods do not make any assumptions about the attacker's strategy, with the only condition being the number of sensors that are corrupted.

\endabstract

\end{frontmatter}
\begin{optionalFrontMatter}
% ===============================================
% OPTIONAL MATERIAL
% Everything after this is optional and can appear in any order
% you desire.
%\begin{dedication}
%% Prints the text of the file dedication.tex centered vertically on the page.
%	\vspace*{\fill} 
%
%	\vspace*{\fill} 
%\end{dedication}
\end{optionalFrontMatter}
\addcontentsline{toc}{chapter}{Contents}

\tableofcontents

\begin{acknowledgements}
\thispagestyle{plain}

First, I would like to take this opportunity to thank my adviser, Professor Claire Tomlin, for her continued support and guidance. Throughout my PhD program, her insight, enthusiasm and dedication to her work never failed to inspire me.
In addition, I would like to thank Professor Anil Aswani for his advice and guidance, especially during the time that I was frustrated and uncertain about my research direction.
Furthermore, for their time and for their input, I thank Professor Murat Arcak for being on my committee and I thank Professor Laurent El Ghaoui for being on my quals committee.
% this thesis would never have been possible without the help of many caring and generous people.

Despite the Hollywood image of a lone scientist, research is not a solitary process. 
Throughout my PhD program, I had the privilege to work with many superb collaborators. Whenever ``we" appears in this thesis, I mean Young Hwan, Dariush, Datong, Vaggelis, Max, Frauke and Sumedh. 
In addition, there are many people that I collaborated with but our work is not included in this thesis: Jeff, Maja, Mo, Kene, Jaime, Casey, Gabriella, Anil, Mark and Soile. 
I learnt a great deal from working with everyone.
I would also like to thank Domenico for facilitating the field experiments on SDH.

In addition, research is not the entirety of a PhD program. I would also like to thank many caring, generous and fun people, who kept my spirit up during the past few years and gave me the courage and strength to finish my PhD: Can, Lucas, Sissi, TJ, Ming, Mo, Jeannette, Anil, Jerry and everyone else I missed.
Finally, I would also like to thank my parents for supporting me in all my endeavors, from quitting my job to go to grad school to moving to a continent far away from home. I would also like to thank them for their help in taking care of Lucas, so that I can even find time to write up this thesis.


\end{acknowledgements}

%%%% END FRONT MATTER...


% ============================= DISSERTATION TEXT
% Begins regular arabic numeral page numbers...
% CHAPTERS

\begin{dissertationText}
\chapter{Introduction \label{chapter:intro}}
	\subfile{chapters/introduction}
\chapter{Mathematical Modeling of Commercial Buildings \label{chapter:building_model}}
	\subfile{chapters/building_model/building_model.tex}
\chapter{Experimental Demonstration of Frequency Regulation from Commercial Buildings \label{chapter:building_exp}}
	\subfile{chapters/building_exp/building_exp.tex}
\chapter{Secure Estimation for Linear Systems under Cyber Attacks \label{chapter:se_linear}}
	\subfile{chapters/se_linear/se_linear.tex}
\chapter{Secure Estimation for Nonlinear Power Systems under Cyber Attacks \label{chapter:se_nonlinear}}
	\subfile{chapters/se_nonlinear/se_nonlinear.tex}
\chapter{Conclusions and Future Work \label{chapter:conclusions}}
  \subfile{chapters/conclusions.tex}



\ssp	% SINGLE SPACE REFERENCES (optional)
\bibliographystyle{plain}
\bibliography{references}


\appendix
\chapter{Proof of Theorem 1 \label{sec:appendix}}
	\subfile{chapters/appendix.tex}

\end{dissertationText}
\end{document}

% Congratulations, Doc!
