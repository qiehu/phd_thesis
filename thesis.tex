\documentclass[12pt, oneside, final]{lib/ucthesis}
\def\ssp{\def\baselinestretch{1.0}\large\normalsize}
\ssp

\usepackage{subfiles}
\usepackage{cite}
\usepackage{array}
\newcolumntype{P}[1]{>{\centering\arraybackslash}p{#1}}
\usepackage{amsmath}    % need for sub equations
\usepackage{amsfonts}
\usepackage{amssymb}
\usepackage{graphicx}   % need for figures
\usepackage{subcaption}
\usepackage{url}
%\usepackage{algorithmic}
\usepackage{algorithm}
\usepackage{algpseudocode}
\usepackage{color}
\usepackage{mathtools}    % need for sub equations
\usepackage{cancel}
\usepackage{bm}
%\usepackage[ruled,vlined,titlenotnumbered,linesnumbered]{algorithm2e}
\usepackage{float}
\usepackage{enumitem}
\usepackage{booktabs}
%\usepackage{gensymb}
%\usepackage{fixltx2e}
\usepackage{array}
\usepackage{bm}
%\usepackage{subfig}

\pdfinfo{
       /Title      (U. C. Berkeley Dissertation)
       /Author     (Qie Hu)
       /Keywords   ()
    }
% This gives you control over how far down in the hierarchy the
% table of contents will print. I use 2.
\setcounter{tocdepth}{1}
\setcounter{secnumdepth}{3}
\setlength{\parindent}{5ex}

% \include{chapters/preamble}

% \graphicspath{{resources/}}

%\newcommand\degrees{\ensuremath{^\circ}}
\newcommand{\tab}{\hspace{5mm}}
\newcommand{\blankpage}{\clearpage ~ \newpage}

\newtheorem{lem}{Lemma}
\newtheorem{remark}{Remark}
\newtheorem{theorem}{Theorem}
\newtheorem{proposition}{Proposition}
\newtheorem{result}{Result}
\newtheorem{claim}{Claim}
\newtheorem{corollary}{Corollary}
\newtheorem{definition}{Definition}[section]
\newtheorem{assumption}{Assumption}[section]

\DeclareMathOperator*{\argmin}{\arg\!\min}
\DeclareMathOperator*{\argmax}{\arg\!\max}

% SE for linear systems
\usepackage{epsfig}
\usepackage{marvosym}
\usepackage{soul}
\usepackage{xcolor}
\usepackage{epstopdf}
\usepackage{cancel}
\usepackage{pifont}
\usepackage{url}
%\usepackage{amsthm}

% SE for nonlinear power systems
\usepackage{steinmetz}

\newcommand*{\dt}[1]{%
\newcommand{\pf} {{\textit{Proof :}}}
%\newcommand{\argmax}{\operatornamewithlimits{argmax}}
\allowdisplaybreaks
\newcommand{\norm}[1]{\left\lVert#1\right\rVert}
\newcommand{\abs}[1]{\left\lvert#1\right\rvert}
 \accentset{\mbox{\large\bfseries .}}{#1}}
%\renewcommand{\baselinestretch}{0.96}
%\newcommand*{\suchthat}{\;\ifnum\currentgrouptype=16 \middle\fi|\;}








\pagestyle{headings}
% ========================================= DOCUMENT
\begin{document}

% Declarations for Front Matter

% TITLE
\title{Toward a Smart and Secure Power Grid}
% Note that this must be exactly as it appears in University records.
\author{Qie Hu}

% PREVIOUS DEGREES
% Put each previous degree on its own line in the following format:
\prevdegrees{} % Optional 

% DATE OF GRADUATION
% This text will appear on the title page
% Note that degrees are only granted in Fall and Spring at Berkeley.
% This text will appear on the abstract page.
% For Berkeley, it should be identical to the graduation month.
\degreeyear{2017}
\degreemonth{Fall}
\defensemonth{Fall}
\defenseyear{2017}

\degree{Doctor of Philosophy}
% COMMITTEE MEMBERS
% You can have up to 5 members listed separately. 
%After that, you throw them all into the "other members" category.
\numberofmembers{3} 
	\chair{Professor Claire Tomlin} 
	\othermemberA{Professor Murat Arcak}
	\othermemberB{Professor Anil Aswani} 
	
% DEPARTMENT/DEGREE PROGRAM
%Your Department. Make sure this is the department and/or program
% name that you are enrolled in...
\field{Engineering - Electrical Engineering and Computer Sciences} 

% CAMPUS NAME
% Your UC Campus, e.g., "Berkeley"
% Note that if you are not at Berkeley, you may have to modify the\vspace{12pt}
% ucthesis.cls to change the wording on the Title page.
\campus{Berkeley}

\begin{frontmatter} 
\maketitle
%\approvalpage
\copyrightpage
\abstract
Automation is becoming pervasive in everyday life, and many automated systems, such as unmanned aerial systems, autonomous cars, and many types of robots, are complex and safety-critical. Formal verification tools are essential for providing performance and safety guarantees for these systems. In particular, reachability analysis has previously been successfully applied to small scale control systems with general nonlinear dynamics under the influence of disturbances. Its exponentially scaling computational complexity, however, makes analyzing more complex, large scale systems intractable. Alleviating computation burden is in general a primary challenge in formal verification.
%$\norm{x}_0$ 

This thesis presents ways to tackle this ``curse of dimensionality'' from multiple fronts, bringing tractable verification of complex, practical systems such as unmanned aerial systems, autonomous cars and robots, and biological systems closer to reality. The theoretical contributions pertain to Hamilton-Jacobi (HJ) reachability analysis, with applications to unmanned aerial system. In addition, this thesis also explores two frontiers of HJ reachability by combining the formal guarantees of reachability with the computational advantages of optimization and machine learning, and with fast motion planning algorithms commonly used in robotics. The potential and benefits of the theoretical advances are demonstrated in numerous practical applications.

\endabstract

\end{frontmatter}
\begin{optionalFrontMatter}
% ===============================================
% OPTIONAL MATERIAL
% Everything after this is optional and can appear in any order
% you desire.
%\begin{dedication}
%% Prints the text of the file dedication.tex centered vertically on the page.
%	\vspace*{\fill} 
%
%	\vspace*{\fill} 
%\end{dedication}
\end{optionalFrontMatter}
\addcontentsline{toc}{chapter}{Contents}

\tableofcontents

\begin{acknowledgements}
\thispagestyle{plain}

First, I would like to express special appreciation ...

\end{acknowledgements}

%%%% END FRONT MATTER...


% ============================= DISSERTATION TEXT
% Begins regular arabic numeral page numbers...
% CHAPTERS

\begin{dissertationText}
\chapter{Introduction \label{chapter:intro}}
%	\subfile{chapters/introduction}
%\chapter{Testbed \label{chapter:testbed}}
%	\subfile{chapters/testbed}
\chapter{Mathematical Modeling of Commercial Buildings \label{chapter:building_model}}
	\subfile{chapters/building_model/building_model.tex}
\chapter{Experimental Demonstration of Frequency Regulation \label{chapter:building_exp}}
	\subfile{chapters/building_exp/building_exp.tex}
\chapter{Secure Estimation for Linear Systems under Cyber Attacks \label{chapter:se_linear}}
	\subfile{chapters/se_linear/se_linear.tex}
\chapter{Secure Estimation for Nonlinear Power Systems under Cyber Attacks \label{chapter:se_nonlinear}}
	\subfile{chapters/se_nonlinear/se_nonlinear.tex}
\chapter{Conclusions and Future Work \label{chapter:conclusions}}
%  \subfile{chapters/conclusions.tex}



\ssp	% SINGLE SPACE REFERENCES (optional)
\bibliographystyle{plain}
\bibliography{references}

%% ------------------- OPTIONAL APPENDICES
%\appendix
%\chapter{Appendix Chapter}~\label{sec:appendix-pf}
%	\subfile{chapters/chapter1}

\end{dissertationText}
\end{document}

% Congratulations, Doc!
