%%%%%%%%%%%%%%%%%%%%%%%%%%%%%%%%%%%%%%%%%%%%%%%%%%%%%%%%%%%%%%%
% introduction.tex (part of thesis.tex)
% author: Qie Hu
%
%%%%%%%%%%%%%%%%%%%%%%%%%%%%%%%%%%%%%%%%%%%%%%%%%%%%%%%%%%%%%%%

% !TEX root = ../../thesis.tex

\subsection{Introduction}\label{sec:introduction}

The operation of a reliable power grid depends on a balance of supply and demand at all times. 
%Any mismatch can cause the grid frequency to deviate from its nominal value (60 Hz in the U.S.).
Reserves known as ancillary services (AS) are used to correct any mismatch.
Amongst them, frequency regulation is the highest quality AS over which the grid operator has almost real-time control and is active continuously during normal operation of the grid, to maintain the grid frequency at its nominal value (60 Hz in the U.S.).
The recent rapid increase in the penetration of renewable energy sources has increased the volatility and uncertainty of electricity supply, which leads to a greater demand for frequency regulation reserves.

Although these reserves have been traditionally provided by conventional power generators, loads on the demand side have also been considered. 
%, by increasing (and decreasing) their electricity consumption when the grid frequency increases (decreases).
In particular, commercial buildings are a tremendous untapped resource for this application. 
First, they account for a large fraction of the total electricity consumption (more than 35\% in the U.S., 39\% of which is due to heating, ventilation and air conditioning (HVAC) systems \cite{USenergy:2013}). 
Second, the building's large thermal capacity allows the power consumption of HVAC systems be partly shifted in time without compromising occupant comfort. 
Third, many commercial buildings are equipped with a variable frequency drive which can be controlled to vary the power consumption of supply fans of the HVAC system quickly and continuously \cite{Hao:2012demandresponse}. This greatly simplifies tracking of the reference regulation signal, as opposed to resources with on-off control.
%as opposed to on-off control of other equipments such as residential air conditioners and electric vehicles.
Fourth, about one third of commercial buildings in the U.S. are equipped with a building automation system (BAS) \cite{Braun:2012} which facilitates the implementation of new controllers.

On the other hand, there are a number of challenges in using commercial buildings for frequency regulation. 
First, obtaining a building model that is amenable to control is not straightforward. 
Because commercial buildings are often subject to large disturbances such as occupancy, that are difficult to capture. 
%In addition, they have many rooms and the rooms may be affected by different types of disturbances. 
In addition, buildings are often not sufficiently excited, as they must satisfy strict regulatory requirements during regular operation, which limit the type and duration of excitation experiments that can be conducted.
Second, about one third of commercial buildings in the U.S. are equipped with variable air volume (VAV) HVAC systems \cite{Hao:2012demandresponse}, which are typically complex with many control variables and interdependent control loops.

%%%%%%%%%%%%%%%%%%%%%%%%%%%%%%%%%%%%%%%%%%%%%

\subsubsection{Related Work}
Most of the existing literature on using commercial buildings for frequency regulation is based on simulation and theory. 
\textit{Zhao et al.} demonstrated, through simulations, that commercial building's HVAC system can provide frequency regulation by adjusting the setpoint of the duct static pressure \cite{Zhao:2013hvac}.
%\cite{Hao:2012demandresponse} and 
In \cite{Hao:2014fan}, the authors showed that %in commercial buildings with VAV HVAC systems, 
up to 15\% of the rated supply fans' power can be used to provide regulation reserves in the frequency range $f \in [1/(3 \text{~min}), 1/(8 \text{~sec})]$. 
In addition, this frequency range can be extended down to $1/(1 \text{~hr})$ if chillers are incorporated \cite{Lin:2013chiller}.
Researchers have also investigated the feasibility of using an aggregation of buildings to provide reserves: \cite{Balandat:2014contractdesign} studied the problem of contract design and \cite{Vrettos:2014aggregation} proposed a hierarchical control scheme for this scenario.

On the experimental side, there has been only a few field tests, and most of them were conducted in controlled laboratory environments with minimal uncertainties.
%\textit{Maasoumy et al.} showed that by adjusting the setpoint of the duct static pressure, the power consumption of the supply fans can be varied quickly without significant changes in the building temperature \cite{Maasoumy:2014exp}.
\textit{Maasoumy et al.} showed that variations in the setpoint of the duct static pressure can indirectly control the supply fans' power \cite{Maasoumy:2014exp}. %, but the authors did not attempt to track any regulation signal \cite{Maasoumy:2014exp}.
In \cite{Lin:2015exp}, experiments were conducted in an unoccupied auditorium where filtered versions of the frequency regulation signal were tracked over 40 minute time durations using two distinct control inputs: the supply fans' speed and the setpoint of the supply air flow rate. 
In their work, the baseline consumption was determined \textit{a posteriori} using a high pass filter, which is not in accordance with the current AS market rules.
Later, \textit{Vrettos et al.} presented a method that computes the reserve capacity and baseline at the beginning of the regulation period \cite{Vrettos:2016flexlab1}, and carried out extensive experiments in single-zone unoccupied test cells \cite{Vrettos:2016flexlab2}.

In another interesting contribution, the power consumption of electric heaters was %varied in a controlled environment 
controlled to provide frequency regulation \cite{Fabietti:2016exp}. 
To minimize uncertainties, experiments were conducted in unoccupied rooms at nighttime.
The same group of researchers then extended these results with a formulation to readjust the baseline in the intraday market to account for prediction errors, and carried out experiments during the day in occupied offices, where indoor climate is controlled using electric heaters \cite{Gorecki:2017exp}.
%a formal method to compute optimal regulation capacities to account for uncertainties and demonstrated their method through a similar set of experiments, but carried out during the day in occupied offices.
Electric heaters have the advantage of being simpler to model compared to VAV HVAC systems and are found in many European buildings.
%therefore they are often chosen by researchers to demonstrate the feasibility of proposed frequency tracking strategies. 
However, since many commercial buildings in the U.S. are equipped with the latter, experimental demonstration using VAV HVAC systems is essential for their wide implementation in the U.S..



%%%%%%%%%%%%%%%%%%%%%%%%%%%%%%%%%%%%%%%%%%%%%

\subsubsection{Contributions and Organization}

In this paper, we experimentally demonstrate that VAV HVAC systems can be controlled to provide frequency regulation using a commercial building in regular operation.
Our contributions are three-fold:

First, we propose a procedure to identify a linear data-driven model of the HVAC system, that is easy to implement with the building in regular operation. 
Our model describes the temperature evolution of multiple building zones and is amenable to control design.
In addition, the model explicitly captures internal gains such as occupancy without the need for additional instrumentation such as carbon dioxide sensors.

Second, we aim to improve existing frequency regulation control algorithms.
%proposed in \cite{Vrettos:2016flexlab1}.
Unlike the frequency tracking controller in \cite{Vrettos:2016flexlab1}, which relies on an accurate supply fan model, we propose a controller that is suitable when the building is subject to larger disturbances and/or modeling uncertainties. 
%Our method also differs from \cite{Lin:2015exp} by computing the regulation capacities and baseline consumption at the beginning of each regulation period.

Finally, on the experimental side, as far as we know, this is the first report where an occupied commercial building equipped with a VAV HVAC system successfully provides frequency regulation.
Experiments are conducted in accordance with Pennsylvania, New Jersey, Maryland's (PJM) certification test (40 minute duration) and tracking requirements (over multiple hours), using historical PJM regulation signals.  
Good tracking performance was achieved despite large disturbances such as occupancy and the use of a simple building model, which demonstrates the robustness of our method to uncertainties.

This paper is organized as follows. We first briefly describe the building where experiments are conducted and our control scheme in Section \ref{sec:problem_statement}. Then, Section \ref{sec:model_id} presents the data-driven identification method for the building and fan models, followed by Section \ref{sec:control}, which describes the controller design in detail. Section \ref{sec:results} then demonstrates the performance our controller through extensive field experiment results.
Finally, Section \ref{sec:conclusions} concludes.
%The paper concludes with the experimental results %and a discussion on the economic potential of utilizing commercial buildings for frequency regulation 
%and conclusions in Sections \ref{sec:results} and \ref{sec:conclusions}.












