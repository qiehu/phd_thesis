%%%%%%%%%%%%%%%%%%%%%%%%%%%%%%%%%%%%%%%%%%%%%%%%%%%%%%%%%%%%%%%
%
% Introduction.tex (part of thesis.tex)
% author: Qie Hu
%
%%%%%%%%%%%%%%%%%%%%%%%%%%%%%%%%%%%%%%%%%%%%%%%%%%%%%%%%%%%%%%%

%!TEX root = ../../thesis.tex

\section{Two-Level Control Scheme}\label{sec:control}


\subsection{High Level Controller}\label{sec:hlc}
\subsubsection{Regulation Capacities}

Let $P_{\text{base}}$ and $u_{\text{base}}$ represent the power consumption of the supply fans and the airflow rate through the fans at the baseline operating point. Then,
\begin{equation}\label{eq:P}
\begin{aligned}
P_{\text{base}}(k) &= h\big(u_{\text{base}}(k)\big)\\
&=h\big(1^\top u(k) + v_{\dot{\text{m}}}(k)\big),
\end{aligned}
\end{equation}
where $u(k) \in \mathbb{R}^6$ represents the controlled airflow rates to the six zones on the fourth floor, and $v_{\dot{\text{m}}}(k) \in \mathbb{R}$ is a disturbance that represents the total airflow rate to the remaining floors.
Now, define $R_{\text{u}}(k)$ and $R_{\text{d}}(k)$ as the up- and down-regulation capacities at time step $k$, respectively\footnote{For demand resources, up-regulation requires a reduction in its power consumption and down-regulation requires an increase in consumption.}. In addition, let $r_{\text{u}}(k)$ and $r_{\text{d}}(k)$ denote the maximum changes in fan speed at time step $k$ as a result of reserve provision. Then, the regulation capacities are as follows

\begin{equation}\label{eq:Ru_Rd}
\begin{aligned}
R_{\text{u}}(k) & = P_{\text{base}}(k) - P_{\text{u}}(k)\\
 & = h\big(u_{\text{base}}(k)\big) - g\big(f(u_{\text{base}}(k)) - r_{\text{u}}(k)\big)\\
R_{\text{d}}(k) & = P_{\text{d}}(k) - P_{\text{base}}(k)\\
 & = g\big(f(u_{\text{base}}(k)\big) + r_{\text{d}}(k)) - h\big(u_{\text{base}}(k)\big),
\end{aligned}
\end{equation}
\noindent
where $P_{\text{u}}$ and $P_{\text{d}}$ denote the fans' power consumption when providing maximum up- and down-regulation capacities, respectively.

During real-time operation, the building receives a normalized real-time regulation signal $\omega \in [-1,1]$ from PJM, which represents the requested regulation amount $R$ as a fraction of the declared capacities $R_{\text{u}}$ and $R_{\text{d}}$. In other words, the building must control its power consumption $P$ at time $k$ to track the following reference (desired) value:

\begin{equation}\label{eq:P_ref}
\begin{aligned}
P_{\text{ref}}(k) &= P_{\text{base}}(k)+R(k) \\
	&= P_{\text{base}}(k)+\begin{cases}
	\omega(k) R_{\text{u}}(k) & \mbox{if } \omega(k) < 0 \mbox{ (up-regulation)} \\ 
	\omega(k) R_{\text{d}}(k) & \mbox{if } \omega(k) \geq 0 \mbox{ (down-regulation).} 
	\end{cases} 
\end{aligned}
\end{equation}

%\noindent Finally, PJM requires that each resource's baseline consumption and regulation capacities be declared at the start of every hour and remain fixed for that hour.

%%%%%%%%%%%%%%%%%%%%%%%%%%%%%%%%%%%%%%%%%%%%%%%%%

\subsubsection{Input Constraints}


The airflow rates to the fourth floor must remain fixed for each hour in order to maintain a constant baseline, in addition to being restricted by the minimum and maximum airflow settings of the HVAC system.
Note that in our experiment setup, $u$ is unaffected by the uncertain regulation signal $w$.
%\begin{equation}
\begin{multline}\label{eq:u_constraint_const}
u(k) = u(k+j) \text{~~for all~} k = 4m+1,~ j = \{1,2,3\}, m \leq n/4-1, ~m \in \mathbb{N},
\end{multline}
%\end{equation} 
\begin{equation}\label{eq:u_constraint}
u_{\text{min}} \leq u(k) \leq u_{\text{max}}.
\end{equation}

Furthermore, there are minimum and maximum fan speed requirements: $N_\text{min} = 20\%$ and $N_\text{max} = 60\%$, where the maximum limit ensures that the HVAC system's duct static pressure remains under the maximum safe value of 2 inches water column, and the minimum limit ensures that the supply fans have sufficient power to drive the supply air throughout the building: 
\begin{equation}\label{eq:N_constraint}
N_\text{min} \leq N(k) \leq N_\text{max} \text{~for all~} \omega(k) \in [-1,1],
\end{equation}
where $N(k)$ is the fan speed at time step $k$.
Since $\omega$ is unknown at scheduling time, the above constraint must hold for any $\omega$. 
The fan model states that $N$ is given by the inverse of the speed-to-power function, $N = g^{-1}(P)$, which is usually complex, however, it has a simple form at the boundary values of $\omega$:
\begin{equation}\label{eq:N_w}
N(k) = \begin{cases}
	N_{\text{base}}(k) - r_{\text{u}}(k) & \mbox{if } \omega(k) = -1 \\ 
	N_{\text{base}}(k) + r_{\text{d}}(k) & \mbox{if } \omega(k) = 1. \\ 
	\end{cases} 
\end{equation}
Thus, the above constraints can be satisfied by imposing the following robustified version of \eqref{eq:N_constraint}
\begin{equation}\label{eq:N_constraint_rob}
N_\text{min} + r_u \leq N_{\text{base}}(k) \leq N_\text{max} - r_d.
\end{equation}
Although \cite{Vrettos:2016flexlab1} showed that the energy content of $\omega$ over a 15 minute interval is typically limited and $\omega \in [ -\omega_{\text{lim}}, \omega_{\text{lim}}]$ most of the time, where $0<\omega_{\text{lim}} < 1$;
we choose to deal with the uncertainty introduced by $\omega$ in a robust fashion, by accounting for the worst case, i.e., $\omega = \pm 1$.
This way, the building need not make any assumptions on the statistical properties of $\omega$, and it has the added advantage of building additional robustness to modeling and forecast uncertainties for an occupied building.

%%%%%%%%%%%%%%%%%%%%%%%%%%%%%%%%%%%%%%%%%%%%%%%%%

\subsubsection{Output Constraints}

The indoor temperature $x$ should be kept within the time varying comfort zone $[x_{\text{min}}(k), x_{\text{max}}(k)]$ for all $k$:
\begin{equation}
x_{\text{min}}(k) \leq x(k) \leq x_{\text{max}}(k).
\end{equation}
Note that $x$ is unaffected by the unknown signal $\omega$ since $u$ is independent of $\omega$.


%Since we choose $u(k)$ independently from the unknown signal $\omega(k)$ and the zone temperatures $x(k)$ are given by \eqref{eq:building_model}, then $x(k)$ is also unaffected by $w(k)$. 
%Therefore to maintain the indoor temperatures within the time varying comfort zone $[x_{\text{min}}(k), x_{\text{max}}(k)]$, we simply impose
%\begin{equation}
%x_{\text{min}}(k) \leq x(k) \leq x_{\text{max}}(k).
%\end{equation}

The regulation capacities are required to be fixed for each hour which translates to the following constraint:
\begin{multline}\label{eq:R_constraint_const}
R_{\text{u}}(k) = R_{\text{u}}(k+j) \text{~and~} R_{\text{d}}(k) = R_{\text{d}}(k+j)\\
\text{~for all~} k = 4m+1,~ j = \{1,2,3\}, m \leq n/4-1, ~m \in \mathbb{N}.
\end{multline}

\noindent In addition, these capacities must be non-negative.
Because fan power is a non-decreasing function of its speed in the operating range $[N_\text{min}, N_\text{max}]$, the following condition ensures $R_{\text{u}}(k) \geq 0$ and $R_{\text{d}}(k) \geq 0$:
\begin{equation}\label{eq:r_constraint} 
r_{\text{u}}(k) \geq 0, ~r_{\text{d}}(k) \geq 0.
\end{equation}

%Furthermore, the regulation capacities must be fixed for each hour and be non-negative, i.e., $R_{\text{u}}(k) \geq 0$ and $R_{\text{d}}(k) \geq 0$. 
%Similar to the constant baseline restriction for each hour, the first requirement can be met by applying the following constraint
%\begin{multline}\label{eq:R_constraint_const}
%R_{\text{u}}(k) = R_{\text{u},k+j} \text{~and~} R_{\text{d}}(k) = R_{\text{d},k+j}\\
%\text{~for all~} k = 4m+1,~ j = \{1,2,3\}, \\
%m \leq n/4-1, ~m \in \mathbb{N}.
%\end{multline}
%To meet the second requirement, we constrain the deviations in the fan speed as a result of reserve provision as follows
%\begin{equation}\label{eq:r_constraint} 
%r_{\text{u}}(k) \geq 0, ~r_{\text{d}}(k) \geq 0.
%\end{equation}
%Since fan power is a non-decreasing function of its speed in the operating range $[N_\text{min}, N_\text{max}]$, the above constraint results in $R_{\text{u}}(k) \geq 0$ and $R_{\text{d}}(k) \geq 0$. 

%%%%%%%%%%%%%%%%%%%%%%%%%%%%%%%%%%%%%%%%%%%%%%%%%

\subsubsection{Cost Function}
The objective of the HLC is to choose the HVAC's operating point to minimize the cost of electricity consumption and maximize rewards from reserve provision. 
%Conceptually, a building should have a higher baseline consumption so that it has more room to reduce its power consumption to provide more up-regulation capacity. Similarly, to provide more down-regulation requires a lower baseline. However, a high baseline consumption also translates to higher electricity cost. Therefore, there is a trade-off.
\begin{assumption}
Electricity cost is calculated based on the baseline consumption.
\end{assumption}
\begin{assumption}
The payment for both up- and down-regulations are the same.
\end{assumption}
Then, the building's objective is to minimize the following cost function
\begin{equation}\label{eq:cost_function}
c(k) P_{\text{base}}(k) - \lambda(k) \big(R_{\text{u}}(k) + R_{\text{d}}(k)\big),
\end{equation}
where $c$ denotes the unit electricity cost and $\lambda$ denotes the reward for providing regulation service.

%%%%%%%%%%%%%%%%%%%%%%%%%%%%%%%%%%%%%%%%%%%%%%%%%

\subsubsection{Robust Optimization Problem}

We introduce the following robust optimization problem:
\begin{equation}\label{eq:opt}
\begin{aligned}
& \underset{u(k), r_{\text{u}}(k), r_{\text{d}}(k)}{\text{minimize}} 
%& & \textstyle\sum_{k=1}^{n} c(k) h(1^\top u(k) + v_{\dot{\text{m}}}(k)) - \lambda(k) (R_{\text{u}}(k) + R_{\text{d}}(k)) \nonumber\\
& & \textstyle\sum_{k=1}^{n} c(k) P_{\text{base}}(k) - \lambda(k) \big(R_{\text{u}}(k) + R_{\text{d}}(k)\big) \\
& \text{subject to}
& & u_{\text{min}} \leq u(k) \leq u_{\text{max}} \\%, \forall k \\
&&& N_{\text{min}} + r_u \leq N_{\text{base}}(k) \leq N_{\text{max}} - r_d \\
&&& x(k) = Ax(k) + Bu(k) + Cv(k) + q(k) \\
&&& x_{\text{min}}(k) \leq x(k) \leq x_{\text{max}}(k)  \\
%& u_{\text{min},i} \leq u_{i}(k) \leq u_{\text{max},i} ~\text{for}~ i \in \mathcal{I}\\
&&& r_{\text{u}}(k) \geq 0, ~r_{\text{d}}(k) \geq 0\\
%&&&u(k) = u_{k+j} \text{~for all~} k = 4m+1,~ j = \{1,2,3\}, \\
%&&& \quad \quad \text{for all}~ k \\
&&& \text{conditions~} (\ref{eq:u_constraint_const}) \text{~and~} (\ref{eq:R_constraint_const}) \text{~for constant }\\
&&& \quad\quad \text{hourly baseline and regulation capacities}.
\end{aligned}
\end{equation}

\noindent (\ref{eq:opt}) is a deterministic nonlinear optimization, with non-convex quadratic cost function and linear equality and inequality constraints. Despite its non-convexity, it can be solved in Matlab using YALMIP \cite{Lofberg:2004yalmip} and the \textit{fmincon} solver in less than one second, due to its small size. The outcome is the baseline operating point for the HVAC system, and the up- and down-regulation capacities $R_{\text{u}}(k)$ and $R_{\text{d}}(k)$ for each time step in the scheduling horizon $k = 1, \ldots, n$.

%%%%%%%%%%%%%%%%%%%%%%%%%%%%%%%%%%%%%%%%%%%%%%%%%

\subsubsection{Disturbance Approximation and Forecast}
The building's dynamics are subject to the following disturbances: ambient temperature $v_{\text{Ta}}$, SAT $v_{\text{Ts}}$ and internal gains due to occupancy $q$. 
Ambient temperature forecasts are obtained from the publicly available database of \textit{darksky.net} \cite{darksky}. 
Analysis of historical data indicates that SAT rarely varies, as a result, it is measured at each time step $k$ and assumed constant for the HLC horizon, i.e., 1 hour for the certification experiment and 4 hours for the tracking experiment.
The internal gains $q$ is first estimated at each time step $k$ from the building model and real-time measurements, and then assumed constant for the HLC horizon.
In addition, the cost function \eqref{eq:cost_function} is affected by $v_{\dot{\text{m}}}$, the total airflow rate to other floors of SDH, through \eqref{eq:P} and \eqref{eq:Ru_Rd}. This is also measured at each time step $k$ and assumed constant for the HLC horizon.




%%%%%%%%%%%%%%%%%%%%%%%%%%%%%%%%%%%%%%%%%%%%%%%%%
%%%%%%%%%%%%%%%%%%%%%%%%%%%%%%%%%%%%%%%%%%%%%%%%%
%%%%%%%%%%%%%%%%%%%%%%%%%%%%%%%%%%%%%%%%%%%%%%%%%


\subsection{Low Level Controller}\label{sec:llc}

The Low Level Controller (LLC)'s responsibility is to vary the fan power $P$ to track the reference $P_{\text{ref}}$. 
Our approach is different from \cite{Macdonald:2014pjm} which used the frequency of the variable frequency drive as the control input, and from \cite{Lin:2015exp} where control inputs, acting as disturbances, are superimposed onto HVAC's existing control loop commands.

We improve on the switched controller proposed in \cite{Vrettos:2016flexlab1}, so that it is suitable for buildings subjected to large disturbances where an accurate fan model is not available. 
It consists of two sub-controllers. 
Controller 1 is a model-based feedforward controller which uses the fan model from \eqref{eq:fan_model} to determine the fan speed required for a given $P_{\text{ref}}(k)$, i.e., $N(k) = g^{-1}(P_{\text{ref}}(k))$. 
This feedforward controller has a fast response and thus, is best suited when there is a large change in the reference regulation signal, i.e., $|P_{\text{ref}}(k) - P_{\text{ref}}(k-1)|>\epsilon$ where $\epsilon$ is a user defined threshold value. 
On the other hand, Controller 1 has non zero steady state error due to inaccuracies in the fan model. 
Therefore, Controller 2 -- a PI controller, is used to reduce any steady state error from Controller 1. 
%In designing PI controllers, there is a trade-off between small overshoot and fast response rate. 
The implementation of the LLC is presented in Algorithm \ref{alg:llc}.
%Consider a small step change in the reference regulation signal, i.e., i.e., $|\Delta P_{\text{ref}}(k) - \Delta P_{\text{ref},k-1}| \leq \epsilon$, Controller 2 is active and is sufficient to provide good tracking.
%On the other hand,  if the step change in the reference signal is large and $|\Delta P_{\text{ref}}(k) - \Delta P_{\text{ref},k-1}|>\epsilon$, then Controller 1 is activated to quickly bring the fans' power consumption close to the required value.
Step 13 is the discrete implementation of the PI controller, where $\Delta t$ is the discretization time step (4 seconds in our case).
In step 15, the fan speed is capped between $N_\text{min}$ and $N_\text{max}$ to satisfy constraint (\ref{eq:N_constraint}).

\begin{algorithm}
\caption{Low Level Controller}
\label{alg:llc}
\begin{algorithmic}[1]
\State Initialize previous tracking error $e(k-1) = 0$, reference regulation signal $P_{\text{ref}}(k-1)=0$ and fan speed $N(k-1)$ 
\While{experiment is running}
	\State Compute baseline fan power $P_{\text{base}}(k) = h(u_{\text{base}}(k))$
	\State Compute reference (desired) fan power $P_{\text{ref}}(k) = P_{\text{base}}(k)+\omega(k) R_{\text{u}}(k)$ if $\omega(k)<0$ and $P_{\text{ref}}(k) = P_{\text{base}}(k)+\omega(k) R_{\text{d}}(k)$ if $\omega(k) \geq0$
	\State Measure actual fan power $P(k)$
	\State Compute current tracking error $e(k) = P_{\text{ref}}(k) - P(k)$ 
	\If {$|P_{\text{ref}}(k) - P_{\text{ref}}(k-1)|>\epsilon$}
		\State Use model-based controller:
    		\State $N(k) = g^{-1}(P_{\text{ref}}(k))$
    	\Else
		\State Use PI controller:
        	\State $N(k) = N(k-1) + K_\text{P} (e(k)-e(k-1))+\frac{K_\text{P}}{K_\text{I}} e(k) \cdot \Delta t$	
    	\EndIf
	\State Cap fan speed: $N(k) = \max\big(\min(N(k),N_\text{max}),N_\text{min}\big)$
	\State Update variables: $e(k-1) \gets e(k)$, $P_{\text{ref}}(k-1) \gets P_{\text{ref}}(k)$ and $N(k-1) \gets N(k)$
\EndWhile
\end{algorithmic}
\end{algorithm}

The improved performance of our switched controller compared to that in \cite{Vrettos:2016flexlab1} is a result of the new switching condition. 
In \cite{Vrettos:2016flexlab1}, the LLC switches between its sub-controllers based on whether the absolute tracking error is greater than a threshold $\bar{\epsilon}$, i.e., $|e(k)|>\bar{\epsilon}$. 
To see why this controller may fail when an accurate fan model is not available, consider a step change in the reference signal $P_{\text{ref}}$ at time step $k$ such that $|e(k)|>\bar{\epsilon}$. 
According to the switching condition in \cite{Vrettos:2016flexlab1}, Controller 1 would become active.
If the fan model is accurate enough such that the fan speed given by the inverse fan model is able to reduce the tracking error at the next time step $k+1$ to $|e(k+1)| \leq \bar{\epsilon}$, then Controller 2 would take over and continue to decrease the tracking error from time $k+1$ onwards.
On the other hand, if the fan model is not accurate enough such that $|e(k+1)| > \bar{\epsilon}$, then the switched controller in \cite{Vrettos:2016flexlab1} is stuck in Controller 1 with a constant fan speed $N(k) = g^{-1}(P_{\text{ref}}(k))$, %unless $P_{\text{base}}(k)$ or $\Delta P_{\text{ref}}(k)$ changes. 
and consequently, a constant tracking error whose absolute value is greater than $\bar{\epsilon}$.
In practice, accurate fan models are often not available for commercial buildings like SDH which are subject to large disturbances and uncertainties. 
Therefore, larger values of $\bar{\epsilon}$ are needed to avoid the above problem. 
We found that for SDH, the required value of $\bar{\epsilon}$ would be so large that the switched controller essentially acts as a simple PI controller, and a PI controller alone is unable to achieve the fast response necessary for a good tracking performance.%frequency regulation signal tracking.

Our proposed switched controller overcomes the above problem and is therefore suitable for buildings where an accurate fan model is not available. Consider the above scenario again, $|P_{\text{ref}}(k+1) - P_{\text{ref}}(k)| = 0 \leq \epsilon$ independent of the fan model, therefore Controller 2 is activated at time step $k+1$ and continues to reduce the tracking error.




%%%%%%%%%%%%%%%%%%%%%%%%%%%%%%%%%%%%%%%%%%%%%%

\subsubsection{PI Controller Tuning}\label{sec:PItuning}
The proportional $K_\text{P}$ and integral gains $K_\text{I}$ of the PI controller are calculated using the %Chien, Hrones and Reswick
Open Loop Ziegler Nichols method \cite{Moderncontrol_book}. Open loop responses of the fan power to step changes in its speed were recorded. 
The delay time and time constant of the response were used to compute the gains $K_\text{P}$ and $K_\text{I}$.
These values served as initial guesses, which we fine tuned later through trial and error. 
The final gain values are: $K_\text{P} = 0.3$, $K_\text{I} = 6.8$.

Note that the Closed Loop Ziegler Nichols tuning method used in \cite{Vrettos:2016flexlab1} may be unsafe for a regular building, because it requires increasing the proportional gain $K_\text{P}$ until the fan power exhibits sustained oscillations, i.e., the system is marginally stable.
Also note that a single PI controller for the entire operating range of the supply fans was found to be sufficient to achieve good tracking performance.















