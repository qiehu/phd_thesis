%%%%%%%%%%%%%%%%%%%%%%%%%%%%%%%%%%%%%%%%%%%%%%%%%%%%%%%%%%%%%%%
% introduction.tex (part of thesis.tex)
% author: Qie Hu
%
%%%%%%%%%%%%%%%%%%%%%%%%%%%%%%%%%%%%%%%%%%%%%%%%%%%%%%%%%%%%%%%

% !TEX root = ../thesis.tex

\documentclass[../thesis.tex]{subfiles}
\begin{document}

\section{Contributions and Organization}

In this paper, we experimentally demonstrate that VAV HVAC systems can be controlled to provide frequency regulation using a commercial building in regular operation.
Our contributions are three-fold:

First, we propose a procedure to identify a linear data-driven model of the HVAC system, that is easy to implement with the building in regular operation. 
Our model describes the temperature evolution of multiple building zones and is amenable to control design.
In addition, the model explicitly captures internal gains such as occupancy without the need for additional instrumentation such as carbon dioxide sensors.

Second, we aim to improve existing frequency regulation control algorithms.
%proposed in \cite{Vrettos:2016flexlab1}.
Unlike the frequency tracking controller in \cite{Vrettos:2016flexlab1}, which relies on an accurate supply fan model, we propose a controller that is suitable when the building is subject to larger disturbances and/or modeling uncertainties. 
%Our method also differs from \cite{Lin:2015exp} by computing the regulation capacities and baseline consumption at the beginning of each regulation period.

Finally, on the experimental side, as far as we know, this is the first report where an occupied commercial building equipped with a VAV HVAC system successfully provides frequency regulation.
Experiments are conducted in accordance with Pennsylvania, New Jersey, Maryland's (PJM) certification test (40 minute duration) and tracking requirements (over multiple hours), using historical PJM regulation signals.  
Good tracking performance was achieved despite large disturbances such as occupancy and the use of a simple building model, which demonstrates the robustness of our method to uncertainties.

This paper is organized as follows. We first briefly describe the building where experiments are conducted and our control scheme in Section \ref{sec:problem_statement}. Then, Section \ref{sec:model_id} presents the data-driven identification method for the building and fan models, followed by Section \ref{sec:control}, which describes the controller design in detail. Section \ref{sec:results} then demonstrates the performance our controller through extensive field experiment results.
Finally, Section \ref{sec:conclusions} concludes.
%The paper concludes with the experimental results %and a discussion on the economic potential of utilizing commercial buildings for frequency regulation 
%and conclusions in Sections \ref{sec:results} and \ref{sec:conclusions}.


\end{document}








