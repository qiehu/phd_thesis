%!TEX root = ../../thesis.tex

\section{Conclusion}
\label{sec:Conclusion}

We identified two state-space models for the thermal behavior of the same multi-zone commercial building using experimental data collected during regular building operation. One of the models is a low-dimensional data-driven model identified using semiparametric regression, the other one is a high-dimensional physics-based resistance-capacitance model. Both models capture the effect of disturbances such as occupancy and electrical appliances that commercial buildings are subjected to, without installation of any additional hardware such as occupancy sensors. 

The identification of both models on the \textit{same building} enabled us to quantitatively compare the performance of these types of models when applied to a real building, which has not been done before. Our results showed that the RMS error of the open-loop temperature prediction of the physics-based model across different thermal zones and temporal seasons is $0.11^\circ \text{C}$ lower than in the data-driven model, a $25\%$ reduction. However, simulating energy efficient MPC schemes under both models suggested both models perform equally well in terms of cost function minimization and constraint satisfaction despite the significantly higher complexity of the physics-based model.

%However is this improvement significant in closed-loop control schemes? To answer this question, we simulated energy efficient MPC schemes for both models on this building, and concluded that the models performed equally well in terms of cost function minimization and constraint satisfaction. However, as expected, the physics-based model was computationally more expensive due to its large number of states and its bilinear nature. 

It is widely known in this field that low-dimensional data-driven models have lower prediction accuracy than high-dimensional physics-based models, and thus have been only proposed for control of less temperature-critical buildings or zones. However, our work investigated an identification method for data-driven models for multi-zone commercial buildings in regular operation and demonstrated that the lower open-loop prediction accuracy of such data-driven models is insignificant in closed-loop control schemes compared to a high-dimensional physics-based model. Based on these findings, we suggest that such data-driven models may be suitable for applications that were previously considered inappropriate, e.g. frequency regulation.

%We identified state-space models for the thermal behavior of SDH with semiparametric regression and a physics-based model on the \textit{same} testbed. The internal gains due to occupants and electric devices were identified for different spatial granularities and different temporal seasons. We found the high-dimensional physics-based model to yield lower estimation errors than the low-dimensional data-driven model due to the inclusion of analytical temperature models based on physical parameters of the building, therefore allowing for higher granularity in temperature predictions. Under an energy efficient MPC scheme, however, both models performed equally well, with the disadvantage of the physics-based model being computationally expensive due to its large number of states that are bilinearly related with inputs.

%We note that the higher fidelity physics-based model should be used for controlling temperature-critical zones in buildings, since it provides higher granularity in addition to higher accuracy. The compact data-driven model, however, is a good alternative for devising a control strategy when less emphasis is put on estimation errors, e.g. at night when occupancy is low. In frequency regulation, the lower-dimensional data-driven model is more suitable for reserve determination as it requires planning over a longer time horizon, whereas the more accurate higher-dimensional physics-based model can be used in reserve provision to maintain the building temperature within comfort bounds and track the frequency regulation signal. Furthermore, while semiparametric regression can be easily applied on any building with a modest requirement of recorded data, the physics-based model requires detailed geometry and construction data about the building, which in practice is often subject to large inaccuracies, and therefore hard to obtain.

%Finally, we are currently working on verifying our hypothesis by designing and implementing a control scheme suitable for frequency regulation, using the data-driven model, into the building operation system of SDH.

