%!TEX root = ../../thesis.tex

\section{Introduction}
\label{sec:Introduction}
%According to \cite{Perez-Lombard:2008aa}, residential and commercial buildings account for up to 40\% of the total electricity consumption in developed countries, with an upward trend. Heating, ventilation and air-conditioning (HVAC) systems are a major source of this consumption \cite{USenergy:2017}. %, ~35% of energy use in commercial buildings is for heating, cooling and ventilation.
%Nevertheless, their power consumption can be flexibly scheduled without compromising occupant comfort, due to the thermal capacity of buildings. As a result, HVAC systems have become the focal point of research, with the goal of utilizing this source of consumption flexibility. From the point of view of energy efficiency, researchers have studied optimization of building control in order to minimize power consumption \cite{Siroky:2011aa, Parisio:2014aa}.  
%More recently, it has been proposed to engage buildings in supporting the supply quality of electricity and the grid stability, by participating in the regulation of electricity's frequency \cite{Balandat:2014contractdesign, Lin:2015exp, Vrettos:2014aggregation, Baccino:2014aa}.
%
%
%All of the above research activities are based on a valid mathematical model describing the thermal behavior of buildings. 

Traditionally, buildings have been modeled with high-dimensional physics-based models such as resistance-capacitance (RC) models \cite{Maasoumy:2014ab, Sun, David, Hao_multizone}, TRNSYS \cite{Duffy:2009aa} and EnergyPlus \cite{Zhao2013EP}. These models are motivated by the thermodynamics of the building and explicitly model the heat transfer between components of the buildings. The advantage of such models is their high granularity of temperature modeling, but a drawback is their high dimensionality which makes them computationally expensive. 
Although there has been extensive work on model reduction, this remains to be a non-trivial task. A large body of this work focuses on linear models, whereas physics-based models for commercial buildings with a VAV HVAC system are bilinear in nature. Furthermore, existing model reduction techniques often result in a loss of interpretability of states \cite{Dobbs:2012aa} and a significant increase in the model's prediction error \cite{Goyal:2012modelreduction}. 

Motivated by these shortcomings, a new direction of research attempts to identify lower-dimensional, data-driven models, e.g. with Input-Output models \cite{Lin:2015exp} and semiparametric regression \cite{Aswani:2012aa}. The purpose is to alleviate the computational complexity in expense for coarser and less accurate temperature predictions.


%The contribution of this paper is two-fold. First, we aim to improve existing
%data-driven model identification techniques. Unlike \cite{Radecki:2012aa}, \cite{Radecki:2013ab}, who model the evolution of the building's energy consumption without a specific control input, we identify a model for temperature evolution in multiple building zones that is amenable to control design, i.e. with airflows as inputs. Our model also differs from that in \cite{Aswani:2012aa}, which uses HVAC supply air temperature as the single control input, resulting in a simpler identification problem but on the other hand, offers less flexibility in control.

In this chapter, we propose both a physics-based method and a data-driven method to identify models of a multi-zone building, that is easy to implement with the building in regular operation, and captures internal gains such as occupancy, without the need of additional instruments like carbon dioxide sensors.
Our procedures use excitation experiments that actively perturb the building and generate data that can be used for more accurate parameter identification.

More importantly, we perform a \textit{quantitative comparison} of the data-driven and physics-based models in terms of open-loop prediction accuracy and closed-loop control strategies, based on the \textit{same testbed} (the entire floor of an office building). % using \textit{experimental data} collected from the building, as opposed to simulated data. 
We conclude that a low-dimensional data-driven model is suitable for building control applications, such as frequency regulation, due to its minor loss of prediction accuracy compared to high-dimensional physics-based models, but significant gain in computational ease. 
To the best of our knowledge, the extant body of literature has analyzed data-driven and physical models for the identification of temperature evolution in commercial buildings only in an isolated fashion (in particular not on the same testbed) \cite{Ma:2011aa}, \cite{Siroky:2011aa}, \cite{Lin:2015exp}, \cite{Qie}. In addition, some of these models were identified for fictitious buildings with synthetic data \cite{Cole:2013aa, Goyal:2013occupancy, David}, while others used experimental data collected under environments with little or no disturbance, e.g. without occupants \cite{Lin:2015exp}. Our work differs from these existing works in two aspects. First, we use experimental data to identify models for a multi-zone commercial building under regular operation, which is subject to significant disturbances such as occupancy. Second, although the existing literature mentions the differences between data-driven and physics-based models, the prevailing isolationist approach does not provide any quantitative comparison with respect to building control applications - a fact we would like to alleviate by juxtaposing a data-driven with a physics-based model.


%The remainder of this paper is organized as follows: In Section \ref{sec:Preliminaries}, we describe the testbed and the experimental data collected for our research. Section \ref{sec:Data_Driven_Model} presents the identification process for a purely data-driven model with semiparametric regression, followed by Section \ref{sec:Physics_Based_Model}, which details the procedure for identifying a physics-based model. Section \ref{sec:Comparison} then compares the performance of the data-driven model and the physics-based model under open-loop prediction accuracy and closed-loop energy efficient optimal control. We show that, despite the higher accuracy of the complex physics-based model compared to the low-dimensional data-driven model, the optimal control strategies with respect to HVAC operation cost while maintaining the thermal comfort of occupants is almost identical for both systems. We conclude in Section \ref{sec:Conclusion} with a summary of our current and intended future work.

